\section{Community: reimagining the IT help desk}

Over the last few decades, digital literacy has emerged as a key competency across all fields of research. Whether you're a meteorologist analysing the numerical data produced by a computer climate model, an historian analysing a digitised archive of political speeches or a medical researcher designing a three-dimensional model of the ankle joint, the success of your research increasingly depends on your computational competency. Since topics like programming and computer aided design are given very little (if any) attention in typical undergraduate programs, this can be a daunting reality for young researchers making the transition into postgraduate studies. The rapid rise of digital research can be equally daunting for well established academics, who never required these skills earlier in their careers.

Recognising digital literacy as a key issue for modern academics, The University of Melbourne decided in late 2013 that it would trial an institutionally-backed support service in this area. As distinct from a typical Information Technology (IT) department devoted to infrastructure maintenance, this new "IT Research Services" department would help researchers with the myraid of research-specific things they do with computers. In order to achieve this ambitious objective, it was quickly realised that a traditional IT help desk was not going to work in this context. In that \textit{reactive} model of providing assistance a problem occurs, a ticket is issued, and then there is a race to solve that problem. This works well for discrete, well defined problems (e.g. "I need my wifi account reactivated" or "I need a new license for Adobe Illustrator"), but in a research context users are solving problems which are exploratory in nature. The researcher typically does not know how to solve their own problem and therefore cannot issue a ticket to get that problem solved. Instead, what is needed is a \textit{proactive} community model, where the knowledge is not retained by a single person or help desk but is instead shared within the community and is continually floating from one person to the next. 


