\section{Community: reimagining the IT help desk}

Over the last few decades, digital literacy has emerged as a key competency across all fields of research. Whether you're a meteorologist analysing the numerical data produced by a computer climate model, an historian analysing a digitised archive of political speeches or a medical researcher designing a three-dimensional model of the ankle joint, the success of your research increasingly depends on your computational competency. Since topics like programming and computer aided design are given very little (if any) attention in typical undergraduate programs, this can be a daunting reality for young researchers making the transition into postgraduate studies. The rapid rise of digital research can be equally daunting for well established academics, who never required these skills earlier in their careers.

Recognising digital literacy as a key issue for modern academics, The University of Melbourne decided in late 2013 that it would trial an institutionally-backed support service in this area. Quite distinct from a typical Information Technology (IT) department devoted to infrastructure maintenance, this new "IT Research Services" department would help researchers with the myraid of things they do with computers.   